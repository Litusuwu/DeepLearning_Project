\section{State of the Art and Baseline}
\label{sec:baseline}

\begin{frame}
    \frametitle{Traditional Approaches to Wildfire Detection}
    \begin{itemize}
        \item \textbf{Sensor-based methods:} Use \textbf{temperature, smoke, and gas sensors}, but have \textbf{limited coverage}.
        \item \textbf{Classic Computer Vision methods:} Use \textbf{color-based segmentation}, but suffer from \textbf{high false positives}.
        \item \textbf{Machine Learning and Deep Learning approaches:}
            \begin{itemize}
                \item CNN-based classification (e.g., Xception, DenseNet, ResNet).
                \item Object detection with \textbf{YOLOv8}.
                \item Vision Transformers (\textbf{ViTs}) for feature extraction.
            \end{itemize}
    \end{itemize}
\end{frame}

\begin{frame}
    \frametitle{Baseline Model: FireSight (Stanford)}
    \begin{itemize}
        \item FireSight combines \textbf{CNNs and ViTs} for aerial wildfire detection.
        \item Achieves \textbf{82.28\% accuracy} using \textbf{DenseNet + ResNet + ViT} ensemble.
        \item Our approach:
            \begin{itemize}
                \item Improve CNN-only ensemble.
                \item Optimize architecture for \textbf{real-time drone deployment}.
            \end{itemize}
    \end{itemize}
\end{frame}