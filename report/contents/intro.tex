\section{Introducción}
Los incendios forestales representan una amenaza para el medio ambiente y la salud pública.
En la actualidad, su impacto ha aumentado por el cambio climático y el crecimiento
de la actividad humana en zonas boscosas \cite{Bot2022}. Estos eventos pueden partir de un origen
natural o antrópico (provocados, por ejemplo, por negligencia humana) y generan
consecuencias graves como daños en infraestructuras, pérdida de biodiversidad y
efectos económicos significativos.

La creciente frecuencia de incendios en bosques y otros biomas ha impulsado el desarrollo
de sistemas automáticos de vigilancia diseñados para detectar fuego y humo de manera
temprana, reduciendo el tiempo de respuesta de los equipos de extinción y minimizando
la propagación de los incendios.

El aprendizaje profundo (\textit{deep learning}) ha demostrado ser altamente eficaz
en el análisis de imágenes, especialmente para la detección de incendios y humo,
superando limitaciones presentes en métodos tradicionales. En este trabajo,
se propondrá el \textbf{desarrollo de un modelo} basado en estas tecnologías con el
objetivo de integrarlo en un dron para la \textbf{detección en tiempo real de incendios
forestales de forma más rápida y eficiente}.

